%----------------------------------------------------------------------------------------
%	TITLE PAGE
%----------------------------------------------------------------------------------------

\title[PADI-IP]{Funções}

% \author{Matheus Pedroza Ferreira} % Your name
% \institute[UFT] % Your institution as it will appear on the bottom of every slide, may be shorthand to save space
% {
% Orientador: Marcelo Lisboa Rocha \\
% \bigskip
% Universidade Federal do Tocantins \\ % Your institution for the title page
% %\textit{matheuspff@gmail.com} % Your email address
% }
\author{Introdução à Prograpação}
\date{29 de Março de 2018} % Date, can be changed to a custom date

\begin{document}

\begin{frame}
\titlepage % Print the title page as the first slide
\end{frame}

\begin{frame}
\frametitle{Conteúdo da Apresentação} % Table of contents slide, comment this block out to remove it
\tableofcontents % Throughout your presentation, if you choose to use \section{} and \subsection{} commands, these will automatically be printed on this slide as an overview of your presentation
\end{frame}

%----------------------------------------------------------------------------------------
%	PRESENTATION SLIDES
%----------------------------------------------------------------------------------------
