\subsection{Estrutura}

\begin{frame}
    \frametitle{Estrutura}
    
        \begin{itemize}
            \item A estrutura de uma função em \textbf{C} é a seguinte:
        \end{itemize}
    
        \begin{equation*}
            \begin{aligned}
                & \textit{tipo do retorno} \quad \textit{nome} \quad ( \textit{parâmetro 1}, \textit{parâmetro 2}, ...,  \textit{parâmetro n} ) \\
                & \{ \\
                & \qquad \textit{código} \\
                & \\
                & \qquad return ...   \\
                & \} \\
            \end{aligned}
            \end{equation*}
        
\end{frame}



\begin{slide}{Estrutura}

    \item Da mesma forma que na definição matemática, uma função possui um conjunto de parâmetros de entrada e uma saída.

    \item Porém, essa função pode ou não retornar um valor.

    \item Cada \textit{parâmetro} consiste em um tipo e um nome de variável.

    \item O \textit{tipo de retorno} da função nos diz o que essa função irá retornar.

    \item Se esse tipo é \textbf{void}, a função não retorna valor algum.

\end{slide}
    



\subsection{Declaração}

\begin{slide}{Declaração}

    \item Em \textbf{C}, podemos criar tanto a \textbf{declaração} de uma função quanto sua \textbf{definição}.

    \item A declaração de uma função indica como a função será chamada e o que ela retornará.

    \item Consiste somente no \textit{tipo do retorno}, \textit{nome}, e no \textbf{tipo} (o nome dos parâmetros é opcional) dos parâmetros.

    \item Pode aparacer diversas vezes no mesmo programa, desde que todas as definições sejam iguais.

\end{slide}



\begin{frame}[fragile]
\frametitle{Declaração}
    
    \begin{itemize}
        
        \item Exemplo de declarações de funções:

    \end{itemize}
    
    \begin{lstlisting}[language=C]
        int funcao1 (int x);
        int funcao1 (int);

        void funcao2 (double);
        
        double funcao3 (int, double, char);
    \end{lstlisting}
    
\end{frame}
    




\subsection{Definição}

\begin{slide}{Definição}

    \item A definição de uma função consiste no corpo da função, o que realmente será executado, assim como a sua declaração.

    \item Pode aparecer somente uma vez no mesmo programa.

    \item A declaração é inclusa nos headers (arquivos .h) e a definição nos arquivos fonte (arquivos .c).

\end{slide}



\begin{frame}[fragile]
\frametitle{Definição}
        
    \begin{itemize}
        
        \item Exemplo de definições de funções:

    \end{itemize}
    
    \begin{lstlisting}[language=C]
        int funcao1 (int x)
        {
            return x * 2;
        }

        void funcao2 (double x)
        {
            printf("%lf\n", x);
        }
    \end{lstlisting}
        
\end{frame}
