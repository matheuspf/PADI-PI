\subsection{while}

\begin{frame}
    \frametitle{while}
    
        \begin{itemize}
            \item A estrutura do \textit{while} é a seguinte:
        \end{itemize}
    
        \begin{equation*}
            \begin{aligned}
                & \bm{while} ( \textit{condição de parada} ) \\
                & \{ \\
                & \qquad \textit{código} \\
                & \} \\
            \end{aligned}
          \end{equation*}
    
\end{frame}
    



\subsubsection{Condição de Parada}

\begin{frame}[fragile]
    \frametitle{Condição de Parada}
    
        \begin{itemize}
            
            \item A condição de parada é executada no início de cada loop, ao contrário do \textbf{for}.

            \item Caso a condição seja verdadeira (qualquer valor diferente de 0), o loop é interrompido.
    
            \item Exemplo:
    
        \end{itemize}
    
        \begin{lstlisting}[language=C]
            int i = 0;
    
            while(i < 10)
            {
                ...     // O 'i' sera modificado aqui
            }
        \end{lstlisting}
    
\end{frame}



\subsubsection{Exemplos}

\begin{frame}[fragile]
    \frametitle{Exemplos}
    
        \begin{itemize}
            
            \item Exemplos de loops \textbf{válidos}:
    
        \end{itemize}
    
        \begin{lstlisting}[language=C]
            while(i < 10)

            while(i >= 10)

            while(i * i < 100)

            while(sin(i) * sqrt(i) + i / 2.0 > 0.3)

            while(i += 1, i < 10)

            while(1)    // Loop infinito
        \end{lstlisting}
    
\end{frame}






\begin{frame}[fragile]
    \frametitle{Exemplos}
    
        \begin{itemize}
            
            \item Exemplos de loops \textbf{inválidos}:
    
        \end{itemize}
    
        \begin{lstlisting}[language=C]
            while(int i = 0)

            while()

            while(0)  // Loop valido, mas nunca 
                         sera executado
        \end{lstlisting}
    
\end{frame}





\subsubsection{Exercícios}

\begin{frame}[fragile]
\frametitle{Exercícios}
    
    \begin{itemize}
        
        \item Encontrar e corrigir os erros do código 'while\_1.c'.

    \end{itemize}
    
\end{frame}