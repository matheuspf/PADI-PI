\subsection{for}

\begin{frame}
\frametitle{for}

    \begin{itemize}
        \item A estrutura do \textit{for} é a seguinte:
    \end{itemize}

    \begin{equation*}
        \begin{aligned}
            & \bm{for} ( \textit{inicialização} ; \textit{condição de parada} ; \textit{incremento} ) \\
            & \{ \\
            & \qquad \textit{código} \\
            & \} \\
        \end{aligned}
      \end{equation*}

\end{frame}



\subsubsection{inicialização}

\begin{frame}[fragile]
\frametitle{Inicialização}

    \begin{itemize}
        
        \item É somente executado somente uma vez, logo no \textbf{início} do \textbf{for}.

        \item É utilizado para inicializar as variáveis que serão utilizadas no loop.

        \item Exemplo:

    \end{itemize}

    \begin{lstlisting}[language=C]
        int i;

        for(i = 0; ... ; ...)
        {
            ...
        }
    \end{lstlisting}

\end{frame}




\subsubsection{Condição de Parada}

\begin{frame}[fragile]
    \frametitle{Condição de Parada}
    
        \begin{itemize}
            
            \item É executada no \textbf{final} de cada iteração.
    
            \item Se seu valor é verdadeiro (qualquer valor diferente de 0), o \textbf{for} é finalizado.

            \item Caso contrário, a próxima iteração continua normalmente.
    
            \item Exemplo:
    
        \end{itemize}
    
        \begin{lstlisting}[language=C]
            int i;
    
            for(... ; i < 10 ; ...)
            {
                ...
            }
        \end{lstlisting}
    
\end{frame}






\subsubsection{Incremento}

\begin{frame}[fragile]
    \frametitle{Incremento}
    
        \begin{itemize}
            
            \item É executada no \textbf{final} de cada iteração, \textbf{após} a condição de parada ser verificada.
    
            \item Usado para alterar o valor de variáveis após cada iteração.
    
            \item Exemplo:
    
        \end{itemize}
    
        \begin{lstlisting}[language=C]
            int i;
    
            for(... ; ... ; ++i)
            {
                ...
            }
        \end{lstlisting}
    
\end{frame}




\subsubsection{Exemplos}

\begin{frame}[fragile]
    \frametitle{Exemplos}
    
        \begin{itemize}
            
            \item Exemplos de loops \textbf{válidos}:
    
        \end{itemize}
    
        \begin{lstlisting}[language=C]
    
            for(i = 0 ; i < 10 ; ++i)

            for(i = 0; i > -10; --i)

            for(i = 2; i < 100; i *= 2)

            for(; i < 100; ++i)

            for(;;)    // Loop infinito

            for(i = 0; ; i++)  // Outro loop infinito
        \end{lstlisting}
    
\end{frame}



\begin{frame}[fragile]
    \frametitle{Exemplos}
    
        \begin{itemize}
            
            \item Exemplos de loops \textbf{inválidos}:
    
        \end{itemize}
    
        \begin{lstlisting}[language=C]
            for(i = 0, i < 10 ; ++i)

            for(i = 0; i > -10)

            for(int i = 2; i < 100; i *= 2)
        \end{lstlisting}
    
\end{frame}



\subsubsection{Exercícios}

\begin{frame}[fragile]
\frametitle{Exercícios}
    
    \begin{itemize}
        
        \item Encontrar e corrigir os erros do código 'for\_1.c'

    \end{itemize}
    
\end{frame}